\documentclass[a4page]{article}
\usepackage{iwona}
\raggedright
\parskip.4\baselineskip
\advance\textheight40mm
\advance\textwidth20mm
\advance\voffset-15mm
\advance\hoffset-6mm
\pagestyle{empty}
\thispagestyle{empty}
%%
\newcommand{\proposal}[1]{\bigskip\textbf{#1}\par\smallskip}
%%
\begin{document}
%\maketitle
\begin{center}
  \large
  Proposals for BSU Third International Week\\
  Tomasz Przechlewski (Kwidzyn/Poland)\\
  tprzechlewski@gmail.com
\end{center}

\vskip3mm
  
\proposal{\#1: Quality statistical charts: introduction to data visualisation}

     The last stage of any serious statistical analysis is publication
     of the results.  This--often neglected or marginally covered at
     most courses on statistics--stage is of a vital importance, as
     poor charts are prone to be misunderstood (infamous `mutant
     numbers'), or worse not attract public attention at all (although
     they are interesting and/or important).  Without doubt quality
     publication is a key to successful promotion.

     As nowadays pictures become ubiquitous means of disseminating the
     results the skills to create clear and high quality statistical
     charts seems to be of the uttermost importance. Yet in reality one can easily
     find infinite examples of bad graphics, often produced by renown
     publisher (Economist journal for example, cf https://www.economist.com/graphic-detail/).

     It this lecture I will show how to produce quality statistical graphics. I will cover
     dot plots, bar charts, lineplots, histograms, kernel density estimates, stripcharts,
     multipanel displays, scatterplots and some others.

     Time: 1.5hr or more

\proposal{\#2: Introduction to reproducible research or how to make statistics more meaningful }
 
     I will introduce a paradigm of reproducibility of research
     (RR)--modern and popular approach to statistical analysis based
     on self-documenting statistical computing (SDS) concept. I will
     try to demonstrate that abandoning traditional trio of
     spreadsheet---editor---presentation program (Excel--Word--PP or Calc--Writer--Impress)
     and switching to
     some modern tools for statistical analysis is a feasible way to
     go, as (some) modern tools are not much more
     difficult that office software while offers greater productivity.

     Using popular OpenSource Statistical Software: R and Rstudio 
     I will demonstrate how to put RR theory into everyday practice.

     Time: 1.0hr or more
     
\proposal{\#3: Statistics from the (true) beginning}

     In the lecture it is argued that one cannot properly understand a
     statistics unless one knows the details of the process by which the numbers
     came into being. It is furthe argued that statistics are
     imperfect (due to fanciful concept definitions, sloppy measures,
     poor samples, and erroneous computations) and one have to
     recognize their limitations.  Yet very often
     courses on statistics (for economics and other
     social sciences in particular) envision it as a branch of
     mathematics, ie. they concentrate on theory, mathematical
     formulas in particular, software commands to compute measure, and
     (to a lesser extent) provide some guidelines how to interpret the
     results.  Lecturers usually focus on clarifying the theory and
     computational complexities while students consider it as boring and difficult.

     As the theory is complicated considerations
     of how real-life statistics came into being (defining concepts,
     measurement, missing data, etc) are usually downplayed, ie. students are presented
     with clean and small ('artificial' or toy) datasets.
     
     I argue that  to properly understand a
     statistics one should start from true beginning (defining concepts,
     measurement, missing data, classifications, etc), and I will illustrate 
     my point with several real-world examples. Data 
     from EuroStat repository (https://ec.europa.eu/eurostat/data/database) will be explained
     and used.

     Time: 1.0hr
     
\end{document}

% Local Variables:
% TeX-master: ""
% mode: latex
% coding: utf-8
% ispell-local-dictionary: "english"
% End:
